\documentclass[article,A4,11pt]{llncs}%
\usepackage[T1]{fontenc}
\usepackage{amsmath}
\usepackage{amssymb}
\usepackage{fullpage, graphicx, url}

\usepackage{epsf,times}
\usepackage{amsfonts}
\usepackage{graphicx}
\usepackage{mathrsfs}
\usepackage{wrapfig}

\usepackage{color}
\usepackage{amsmath,mathrsfs,bm}
\usepackage{cases}
\usepackage{subfig}
\usepackage{multicol}
\usepackage{tabularx}
% DIMIER
\usepackage[T1]{fontenc}
%\newcommand{\tmname}[1]{\textsc{#1}}
%\newcommand{\tmop}[1]{\ensuremath{\operatorname{#1}}}
%\newcommand{\tmsamp}[1]{\textsf{#1}}
%\newcommand{\tmtextsc}[1]{{\scshape{#1}}}
%\newcommand{\tmtextsl}[1]{{\slshape{#1}}}
%\newcommand{\tmtexttt}[1]{{\ttfamily{#1}}}



\leftmargin=0.2cm
\oddsidemargin=1.2cm
\evensidemargin=0cm
\topmargin=0cm
\textwidth=15.5cm
\textheight=21.5cm
\pagestyle{plain}
\setlength{\columnsep}{20pt}


%\def\m{\mathbf{m}}
%\def\H{\mathbf{H}}
%\def\E{\mathbf{E}}
\newcommand{\vepsi}{{\varepsilon}}
\def\hnorm#1#2{\vert\,#1\,\vert_{#2}}
\newcommand{\R}{{\mathbb R}}
\newcommand{\Sph}{{\mathbb S}}
\def\x{\mathbf{x}}
\def\hvec{\overline{\mathbf{h}}}
\def\evec{\overline{\mathbf{e}}}

\DeclareMathAlphabet{\mathpzc}{OT1}{pzc}{m}{it}
%\leftmargin=0cm
%\oddsidemargin=1cm
%\textwidth=14cm
%\pagestyle{plain}

\newcommand{ \etal}{\mbox{\emph{et al. }}}

\newcommand\vect[1]{\mbf{#1}}
\newcommand{\mbf}[1]{\mbox{\boldmath$#1$}}
\newcommand{\RC}[1]{#1 $\times$ #1 $\times$ #1}
\def\um{$\mu$m}
\def\C{$^{\circ}\mathrm{C}$}

\def\clovek#1{\noindent\bgroup\vbox{\noindent#1}\egroup\vskip1em}

% DEFINITION OF CUSTOM FONT SIZE
\newcommand{\customfontA}{\fontsize{50}{55}\selectfont}
\newcommand{\customfontB}{\fontsize{14.4}{20}\selectfont}
\newcommand{\customfontC}{\fontsize{30}{35}\selectfont}

% TO INPUT BACKGROUND IMAGE
\usepackage{eso-pic}
\newcommand\BackgroundPic{
\put(0,0){
\parbox[b][\paperheight]{\paperwidth}{%
\vfill
\centering
%\includegraphics[width=\paperwidth,height=\paperheight]{background_plzen.jpg}%
%\includegraphics[width=\paperwidth,height=\paperheight]{background_tmp.jpg}%
\vfill
}}}

% BEGIN DOCUMENT
\begin{document}

% inputting background image
%\AddToShipoutPicture{\BackgroundPic}

\vbox{}
\pagestyle{empty}

\newpage

\textwidth=15.5cm

\ClearShipoutPicture

\newpage

\section*{}%

\vspace*{60mm}
%ISBN 978-80-7043-898-5\\ \\
This is a joint publication of the University of Nevada (Reno, USA),
University of West Bohemia (Pilsen, Czech Republic),
Czech Technical University (Prague, Czech Republic),
Institute of Thermomechanics (Prague, Czech Republic), and
FEMhub Inc (Reno, USA).\\

\noindent
FEMTEC 2013 \\
3rd European Seminar on Computing\\

\noindent
\begin{tabular}{ll}
Editors: & Pavel Solin (University of Nevada, Institute of Thermomechanics) \\
 & Pavel Karban (University of West Bohemia) \\
 & Jaroslav Kruis (Czech Technical University) \\
Publisher: & University of West Bohemia \\
 & Univerzitn\'{i} 8, 306 14 Plze\u{n}\\
 & Czech Republic\\
Printed by: & Dragon Print, s.r.o \\
 & Klatovsk\'{a} 24, 301 00 Plze\u{n}\\
 & Czech Republic\\
Year: & 2012\\
\end{tabular}

\subsection*{Contact Information}

Mailing address:\\
FEMTEC 2013 Conference\\
FEMhub Inc.\\
5490 Twin Creeks Dr.\\
Reno, NV 89523\\
U.S.A.

\noindent
E-mail: {\tt femtec2013@femhub.com}\\
Web page: {\tt http://femtec2013.femhub.com/}\\
Phone: 1-775-848-7892

\chapter*{\huge FEMTEC 2013}
\vspace{-5mm}
\normalsize
\begin{center}
3rd European Seminar on Computing,
Pilsen, Czech Republic,
June 25 - 29, 2012
\end{center}
\vspace{-3mm}

\section*{Main Thematic Areas}%

Multiphysics coupled problems; Higher-order computational methods
Computing with Python; GPU computing; Cloud computing.

\section*{Application Areas}%

Theoretical results as well as applications are welcome. Application areas include, but are not limited to: Computational electromagnetics, Civil engineering, Nuclear engineering, Mechanical engineering, Nonlinear dynamics, Fluid dynamics, Climate and weather modeling, Computational ecology, Wave propagation, Acoustics, Geophysics, Geomechanics and rock mechanics, Hydrology, Subsurface modeling, Biomechanics, Bioinformatics, Computational chemistry, Stochastic differential equations, Uncertainty quantification, and others.

\subsection*{Scientific Committee}%

%\hspace{4mm}

\begin{itemize}
\item Valmor de Almeida (Oak Ridge National Laboratory, Oak Ridge, USA)
\item Zdenek Bittnar (Faculty of Civil Engineering, CTU Prague)
\item Alain Bossavit (Laboratoire de Genie Electrique de Paris, France)
\item John Butcher (Auckland University, New Zealand)
\item Antonio DiCarlo (University Roma Tre, Rome, Italy)
\item Ivo Dolezel (Czech Technical University, Prague, Czech Republic)
\item Stefano Giani (University of Nottingham, UK)
\item Glen Hansen (Sandia National Laboratories, Albuquerque, USA)
\item Pavel Karban (University of West Bohemia, Pilsen, Czech Republic)
\item Darko Koracin (Desert Research Institute, Reno, USA)
\item Dmitri Kuzmin (University of Erlangen-Nuremberg, Germany)
\item Stephane Lanteri (INRIA, Sophia-Antipolis, France)
\item Jichun Li (University of Nevada, Las Vegas, USA)
\item Shengtai Li (Los Alamos National Laboratory, Los Alamos, USA)
\item Alberto Paoluzzi (University Roma Tre, Rome, Italy)
\item Jean Ragusa (Texas A\&M University, College Station, USA)
\item Francesca Rapetti (University of Nice, France)
\item Sascha Schnepp (Technical University of Darmstadt, Germany)
\item Stefan Turek (Technical University of Dortmund, Germany)
\end{itemize}

\subsection*{Organizing Committee}

\begin{itemize}
\item Pavel Solin (University of Nevada, Reno \& Institute of Thermomechanics, Prague)
\item Pavel Karban  (University of West Bohemia, Pilsen)
\item Jaroslav Kruis (Czech Technical University, Prague)
\end{itemize}

\newpage
{\ }

\tableofcontents

%%%%%%%%%%%%%%%%%%%%%%%%%%%%%%%%%%%%%%%%%%%%%%%%%%%%%%%%%%%%%%%%%%%%%%%%%%%%%%%%%%%%%%%%%%%%%%%%%%%%%%%%%%%%%%%%%%%%%%%%%%%%%%%%%%%%%%%%%%%%%%%%%%%%%%%%%
\part{Abstracts of Keynote Lectures}

\pagestyle{plain}
